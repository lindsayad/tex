% Authors: Alexander Lindsay
% License: Creative Commons Attribution 4.0 International
% License link: creativecommons.org/licenses/by/4.0/legalcode
% Contact: adlinds3@ncsu.edu
%
\documentclass[12pt]{article}

\usepackage{amsmath}
\usepackage{verbatim}
\usepackage{mathtools}
\setlength{\parindent}{0pt}
\setlength{\parskip}{2ex plus 0.5ex minus 0.2ex}
%% \usepackage[none]{hyphenat}

\begin{document}

\textbf{Seminar Abstract}

Open, available simulation tools are sorely lacking in the low-temperature
plasma community. Researchers are constrained to use expensive, proprietary
tools like Comsol or old academic codes that do not have any notion of modern
software development practices such as version control. In lieu of the absence
of available tools, the plasma application Zapdos, built on top of
the MOOSE finite element framework, has been developed. Established capability
in Zapdos focuses on self-consistent solution of the continuity equations for
charged particles, an electron energy equation that governs rates of
electron-impact processes, and the Poisson equation for potential. The
application of these equations to analysis of physics at the interfaces of
plasmas and liquids is presented. Additionally, some preliminary results on
solution of electromagnetic problems such as propagation through a waveguide are
shown. The latter work is in preparation for expansion of Zapdos to fusion
plasmas.

In a similar vein, the space of simulation tools in the burgeoning field of
molten salt reactors is largely vacant. The new MOOSE application Moltres
implements equations for coupling neutron diffusion, core temperature, and
precursor concentrations. Some initial k-eigenvalue and transient results are
presented. The talk concludes by exploring recent expansion of the python
visualization package yt to support unstructured meshes, including natural
rendering of quadratic elements. The work supports easy generation of
publication quality figures from MOOSE application results as demonstrated by a
couple of example scripts.

\end{document}

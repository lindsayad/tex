\documentclass[11pt]{article}

\usepackage{setspace}
\usepackage[top=1in, bottom=1in, left=1in, right=1in]{geometry}

\setlength{\parindent}{0pt}
\setlength{\parskip}{2ex plus 0.5ex minus 0.2ex}

\begin{document}
\onehalfspace
\textbf{Momma's birthday present: December 2014}

Hi Momma, here is your birthday present! I really liked your request for this year. In case you're interested, this document was prepared using software called LaTeX. It's open source, which is part of what I will be talking about in this letter.

There are two parts to my job as a graduate student researcher. I view my job as a combination of work/play. I view it this way because I believe that you've only found the right career if it feels like play. The two parts of my job are the normative and the positive. The positive refers to what my current job is. The normative refers to what I want my job to be. I study the interaction of atmospheric plasmas with water. I will expound more on what those terms mean later in this document. I am studying these plasma-water interactions using computer models/software. Currently (the positive) I am using programs called Comsol and Mathematica to model our real-world plasma-liquid experiments. Comsol and Mathematica have helped me come up with some great results; I presented some of them at my conference in November, and I am currently writing a journal article that will be submitted in February. I will continue to use Comsol and Mathematica as long as they are the most time-efficient and accurate means of modeling our plasmas because I want to graduate as quickly as possible and because my advisors care a lot more about generating results than they do about how those results are generated. Their perspective is totally understandable. However, in my ideal world I would like to be generating our model results using open-source software. Comsol and Mathematica are proprietary. What this means is that the company that designs the proprietary software supplies its customers only with binary applications, e.g. files that are written purely in machine language as opposed to language that humans can understand. The customer/user (you and me) can only run the applications; the user cannot modify the application. Applications that you are familiar with are Microsoft Word, your email program (Microsoft Outlook), and your internet browser (you and Dad use Google Chrome). These applications are all proprietary. There is absolutely nothing ethically wrong with using proprietary software. The software is useful to us and so we buy it and use it, just as we would buy any other useful good/service. However, currently being a member of the academic community, which is founded on curiosity and exploration of basic theories and concepts, I feel like we should demand the freedom to access the source code which underlies the software that we use. Virtually all of the ideas that company programmers use to create Comsol and Mathematica come from the basic research that the computer science/software engineering academic community has conducted. It makes little sense to me why we acadademics generate these ideas, publish them in the open public domain, and then turn around and purchase our own ideas in a closed restricted form! This is not an indictment of the Comsol and Mathematica programmers; it's an indictment of us academics! We freely share our ideas but don't demand a free sharing of ideas in return! What are we doing?!  If we academics openly shared and used our own sofware, we would dramatically decrease our cost of research and increase our productivity! Thus, while at work I will continue to use Comsol and Mathematica because those are the tools that I currently know best (the positive), in my down time I am exploring open-source alternatives (the normative). If I could conduct all my work with open-source software, then I would be very thankful. My above thoughts are probably disjointed and jumbled; though I've shared them orally with several people, including Jessie L., this is the first time I've ever committed them to paper. It seemed appropriate that I do it in a letter to Momma!

Anyways, I'll tell you know what I am trying to model, whether it be with proprietary or open-source software. I am trying to model the interactions of atmospheric plasmas with water. Let me tell you a little bit about plasmas. Plasmas are all around us; examples of plasmas are the sun, lightning, and fluorescent lights. How do you make a plasma? What makes it a plasma? A plasma is an ionized gas. I'm going to explain all the terms I use assuming you don't know what they mean...you may very well already know what some of them mean! Air is an example of a gas. Air is composed of 80\% nitrogen molecules and 20\% oxygen molecules. Molecules are the building blocks of the world. Molecules are composed of different elements bonded together with clouds of electrons swirling around the nuclei of the elements. You can think of a nucleus as the sun and electrons as the planets that move and rotate around the sun. When you ionize a molecule, you pull an electron off of it. Imagine pulling Mars out of its orbit around the sun and sending it spinning it off on its own. Once the electron is free of the molecule, it can move under the influence of an external electric field. That is how we generate electricity! You apply an electric field, and then the electrons flow from the power plant to yor home! Have you ever seen a video of someone holding a paper clip up to a power outlet? When the paper clip gets close enough you see a bright blue arc between the outlet and the paper clip. That's a plasma! Electrons pulled off of air molecules are flowing between the outlet and the paper clip. This occurs because there's a pretty large electric potential at the outlet (120 Volts; you may be familiar with that number) relative to the paper clip. The larger the potential difference between the two metal surfaces (the paper clip and the metal within the power outlet), the larger the electric field in the air and the easier it is to pull electrons off the air molecules. That is very similar to how we generate plasmas (like that bright blue arc between the outlet and the paper clip or the lightning bolt we see coming from the clouds); we create a very large electric field between two metal surfaces that leads to a breakdown of the air molecules (pulling the electrons off of the air molecules). After breakdown, we have our plasma! What makes plasmas so cool? Well, once we've torn the electrons off their host molecules, then they race around in that high electric field we created between the two metal surfaces. Racing around in that high electric field they can acquire a lot of energy; with that energy they can collide with other molecules and create interesting new molecules! Some of these new molecules (with labels like OH, NO, NO$_2$, etc.) can be very useful for different real world applications. Some of these applications include creating fertilizer or treating cancer or wounds! Cancer and other bad things in the body are known to hate molecules like OH. Thus we want to make OH with our plasmas and then blast the cancer or other bad things with it! Unfortunately, in order to get our OH molecules to the bad things in the body, it often has to be transported through a layer of water. That is what our Berkeley group and I are studying: how molecules generated in the plasma are transpored from the plasma into the water. There are some very interesting physics and chemistry that occur where the plasma and the water meet. That's my research: studying those interactions! Hopefully I've given you a decent layman's summary. It's hard to explain it all without using some big, sciency words. Let me know if you want any more explanation.

I love you Mom! Happy birthday!

\end{document}

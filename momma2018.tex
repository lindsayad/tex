\documentclass[11pt]{article}

\usepackage{setspace}
\usepackage[top=1in, bottom=1in, left=1in, right=1in]{geometry}
\usepackage{graphicx}

\setlength{\parindent}{0pt}
\setlength{\parskip}{2ex plus 0.5ex minus 0.2ex}

\begin{document}
\onehalfspace
\textbf{Momma's birthday present, December 2018: Letter III}

Hi Mom! My last two letters seem to have been pretty philosophical, but this one
should be pretty straightforward and focused on the contents of work. I have
been at INL for a year and a half now working on the Multiphysics Object
Oriented Simulation Environment (MOOSE). MOOSE is used for solving equations
related to physics and engineering. It's used to try and model real world
processes to replace the need for costly experiments. I'll try and explain what
specific work I have been doing to advance MOOSE.

My biggest project in the last year has been on something called automatic
differentiation. In order to solve our differential equations we use a process
called Newton's method, which I won't go into the details of. But essentially in
order to effectively use Newton's method, we have to be able to take
derivatives of our equations with respect to our unknowns. Derivative is a fancy
word for rate-of-change. We can say that speed is the time-derivative of
position, e.g. a car's speed is the time derivative of the car's position. It
can be very difficult to calculate derivatives; it's a very error-prone process
and can lead to very slow simulations. So my co-worker Derek, another developer
named Roy, and I created automatic differentiation capabilities in
MOOSE. Instead of having humans calculate the required derivatives, we now let
the computer do it! The machinery behind automatic differentiation involves
fairly complex C++ template programming; getting that to work was a very fun
learning experience. (C++ is the programming language behind MOOSE. It's a very
universal language that companies like Google, Facebook, and Microsoft all use.)

I would say that my second largest contribution to MOOSE is introducing what we
call ``vector'' finite elements. I'll give a brief definition of both ``vector''
and ``finite element.'' Let's say I want to solve for the motion of water in a
swimming pool, e.g. I want to be able to predict how the water will move after a
kid jumps into the pool. In order to get an accurate prediction using a
computer, I need to break up the solution domain (the swimming pool) into
individual pieces, what we call ``finite elements'', and then solve for the
water motion in each small individual piece. I could break the pool up, for
example, into a bunch of 1 $\rm{ft}^3$ water blocks. Until last year, MOOSE
could only solve for ``scalar'' variables on these finite elements. The
difference between a ``scalar'' and a ``vector'' is that the latter has a
direction. Acceleration is an example of a vector; in the car you can accelerate
forwards or you can accelerate backwards. You can accelerate to the left or to
the right. Essentially if I can ask where something is pointing, then that thing
is a vector. Where does gravity point? It points down. It's a vector. What
direction does the earth's magnetic field point? It points towards the
south pole; it's a vector. Things that aren't vectors (so they're scalars):
temperature, weight, pressure, brightness. So I lied a little bit; before last
year, we could actually solve for vectors, but we could only do it by
decomposing them into scalars. We would split acceleration for example into
three pieces: left and right, top and bottom, front and back. This is actually a
clumsy thing to do: it forces people who use MOOSE to write more code than they
need to, and any time people have to write more code, there are greater chances
for mistakes. Moreover, there are actually more elegant ways to solve vector
problems (like in electromagnetism!) that one can use when legitimately treating
vectors like vectors and not as individual scalars. So last winter I added this
vector capability into MOOSE and now people like my friend Casey are using that
capability to model things like radio antennas!

So the last two paragraphs were about very mathy topics. I'll touch a little now
on the more ``physicy'' things that I've been doing. One project I've been
working on is focused on ``mechanical contact.'' ``Mechanical contact'' is a
fancy way of referring to the interactions that take place when you make two
objects physically touch each other. So in a nuclear reactor for example, the
fission process heats up the nuclear fuel and it slowly expands until it touches
the surrounding structural material (basically just metal tubes). We want to
understand the forces that occur when these materials come into contact because
we want to ensure we don't place too much strain on our structures; we don't
want them to bend and break! Resolving mechanical contact is actually the most
mathematically difficult problem we solve with MOOSE. The equations governing
mechanical contact are not ``smooth.'' So my friend Fande and I have been
looking at how to solve the mechanical contact problem in a more mathematically
rigorous way using a concept called ``variational inequalities''. I won't go
into the details about the latter topic, but I'll just say that it's been a very
interesting project and it's not yet done! I do hope to wrap up my contributions by
the end of the next fiscal year (September 2019).

The other ``physicy'' thing I've been working on is laser welding. You probably
know the concept of welding just as well as I do, but it's essentially the
process of joining two solids by melting them together and then allowing the
result to cool and solidify. Laser welding is a pretty neat physical process. To
model it we have to capture several things. We need to be able to capture the
transition from solid to liquid, and the resulting flow patterns of the liquid
which depend on evaporation of material from the liquid pool, surface tension
effects (these are like the forces that keep a bead of water intact on a
Gore-tex jacket), and other factors. Capturing the deformation of the liquid
weld pool due to some of those forces is actually a very challenging modeling
process, but the results look really neat! I can definitely show you a video
sometime! I'll note that some of the things I've done to make these laser
welding simulations work will be applicable to other engineering disciplines
such as fluid-structure interaction which looks at things like movement of blood
capillaries according to the motion of blood through the body.

We'll that's a re-cap of the most important work I've been doing with MOOSE since
starting employment at INL. The next few months will involve finishing the
mechanical contact work and supporting our user base as they begin adopting
automatic differentiation. It's difficult to talk about what I do without using
some technical phrases, but hopefully I've minimized that as much as possible or
given examples in the case where they must be used.

Merry Christmas Mom!

Love, Alex

\end{document}

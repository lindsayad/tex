\documentclass[12pt]{letter}

\usepackage[top=1in, bottom=1in, left=1in, right=1in]{geometry}
\usepackage{url}
\usepackage[none]{hyphenat}

\begin{document}

\signature{Dr. Alexander Lindsay\\ INL Computational Scientist\\\url{alexander.lindsay@inl.gov}}
%\date{DATE}

\begin{letter}{University of Illinois\\ Department of Nuclear, Plasma, and
    Radiological Engineering}
\opening{Selection committee,}

I highly recommend Gavin Ridley as a candidate for graduate school. I had the
opportunity to mentor Gavin this summer and I can testify that even as an
undergraduate his skill, drive, and passion for research surpass those of many senior
graduate students.

Gavin is receiving his undergraduate degree in nuclear engineering from the
University of Tennessee, Knoxville, widely regarded as one of the top ten
programs in the country. While working through the program he has maintained an
impressive 3.89 GPA while conducting meaningful research under Ondrej Chvala, a
leader in the study of molten salt reactors (MSRs). Gavin's research has
produced a secondary authorship as well as a primary authorship, a rare
accomplishment for an undergraduate and a reflection of his passion for the
nuclear engineering field.

In addition to obtaining results and producing publications, Gavin sets himself apart
in his broad participation in reactor physics and computational
science communities. Gavin is an active contributor to discussions on Serpent,
OpenMC, and MOOSE forums. (The former two packages are for Monte Carlo simulation
of neutronics; the latter is a general purpose finite element multiphysics
simulation tool.) He is ahead of the curve in adopting modern software
engineering practices and applying them to tools for computation in nuclear
disciplines. Consequently, he is proficient in UNIX command line tools, version
control, and continuous integration testing, all increasingly important piecies
in the development of sustainable research codes. Too often a computational
graduate student receives his Ph.D. and leaves an opaque, nearly
incomprehensible tool behind for the next poor student to puzzle through; with
Gavin's modern computational practices, attention to detail, and devotion to
transparent research, the aforementioned case is not a concern.

Gavin reached out to me last year and expressed interest in a molten salt
reactor package I was developing called Moltres. After meeeting him at a MOOSE
workshop and observing his personability, ability to quickly pick-up a
sophisticated object-oriented C++ code, and incapability to suppress babbling on
about advanced reactor designs (he is an unstoppable fountain of ideas), I knew
we had to hire him on for the summer. His love of reactors is contagious; as
soon as he arrived my own productivity increased, and together we were able to
develop both steady-state and transient accident scenarios for MSRs. The
steady-state portion of the work has been accepted for publication and Gavin is
currently working on the manuscript for the transient cases. Realizing two
publications in the span of two months is an impressive demonstration of Gavin's
productivity. Gavin continues to develop for Moltres, and I hope to continue
collaborating with Gavin indefinitely.

Having worked closely with Gavin (literally adjoining desks for two months), I
can say with complete confidence that not only will Gavin be a successful
graduate student, but also a career leader in advanced reactor computational
research. His talent combined with his love for the field make this
indisputable. I cannot recommend Gavin more highly; any graduate institution
will be lucky to have him and subsequently claim a piece of his following accomplishments.

\closing{Sincerely,}
\end{letter}


\end{document}
